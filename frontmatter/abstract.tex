%%%%%%%%%%%%%%%%%%%%%%%%%%%%%%%%%%%%%%%%%%%%%%%%%%%%%%%%%%%%%%
% abstract.tex                                               %
% This file formats and includes the content of the abstract %
%%%%%%%%%%%%%%%%%%%%%%%%%%%%%%%%%%%%%%%%%%%%%%%%%%%%%%%%%%%%%%

\cleardoublepage
\thispagestyle{empty} % no page numbers on this page

\chapter*{Abstract} % make this a \chapter* to not include it in the table of contents

\noindent Particle swarm optimization has developed as a popular method of solution 
discovery for many numerical problems. The finite thrust transfer problem employs
this metaheuristic algorithm for trajectory optimization.  This method is computationally expensive and requires
numerous runs to confidently discover a quasi-optimal solution. These major bottlenecks - execution time and 
a low probability of optimal solution convergence - make real-time, on-board calculations utilizing this algorithm 
impractical. \newline

\noindent The research presented within this thesis studies two methods to significantly improve each of these bottlenecks.
This thesis develops results that show increased algorithm performance of well over an order of magnitude, and demonstrably
improves the average solution discovered. In employing these two methods in concert, this
research suggests the potential feasibility of real-time particle swarm optimization techniques.\newline

\cleardoublepage
