%%%%%%%%%%%%%%%%%%%%%%%%%%%%%%%%%%%%%%%%%%%%%%%%%%%%%%%%%%%%%%
% abstract.tex                                               %
% This file formats and includes the content of the abstract %
%%%%%%%%%%%%%%%%%%%%%%%%%%%%%%%%%%%%%%%%%%%%%%%%%%%%%%%%%%%%%%

\cleardoublepage
\thispagestyle{empty} % no page numbers on this page

\chapter*{Abstract} % make this a \chapter* to not include it in the table of contents

\noindent Particle swarm optimization has developed as a popular method of solution 
discovery for many numerical problems. Yet, these methods are computationally expensive and require
numerous runs to confidently discover a quasi-optimal solution. These major bottlenecks - execution time and 
a low probability of optimal solution convergence - restrict the usage of particle swarm methods for time-sensitive calculations. 
In an exploration of the finite thrust arc problem as a benchmark case, this paper analyzes particle swarm
optimization improvements in an effort to broaden the algorithm's utility.\newline

\noindent The research presented within this thesis studies two methods to significantly improve each of the major particle swarm bottlenecks.
First, reducing execution time is explored through various means of problem implementation. 
Memory-optimized single-threaded and parallelized C++ algorithms
are found to perform up to 96\% faster then MATLAB. 
Second, minimizing premature convergence is studied with \textit{rehydration} - a proposed method of resetting a portion of the
swarm under population stagnation. 
\textit{Rehydration} results demonstrate up to a 44\% improvement in average solution discovery. In employing these two methods in concert, this
research suggests the potential feasibility of reliable, real-time particle swarm optimization techniques.\newline

\cleardoublepage
