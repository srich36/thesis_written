%%%%%%%%%%%%%%%%%%%%%%%%%%%%%%%%%%%%%%%%%%%%%%%%
% chapter3.tex                                 %
% Contains formatting and content of chapter 3 %
%%%%%%%%%%%%%%%%%%%%%%%%%%%%%%%%%%%%%%%%%%%%%%%%
\chapter{Problem Implementation}
\section{Overview}
\noindent The finite thrust arc problem consists of 11 unknowns and requires two numerical integrations for each solution candidate within the
particle swarm. Additionally, it necessitates a high absolute and relative tolerance when integrated with variable step sized integrators. These
components serve to make this problem a suitable benchmark for the performance of different implementations of the particle swarm optimization algorithm. \newline

\noindent To maintain consistency between different implementation targets, the numerical integration algorithm, absolute and relative tolerance values,
and minimum step size constraints were held constant. All implementations employed a fifth order Runge-Kutta Dormand Prince variable step size integrator
throughout both thrust arcs. The tolerances were defined as

\begin{equation}
AbsTol = 10^{-9} \quad \text{and} \quad RelTol = 10^{-9} 
\label{eq:tolerances}
\end{equation}

\noindent And the minimum step sized as 

\begin{equation}
    h_{min} = 7.105427*10^{-15}
    \label{eq:min-step-size}
\end{equation}

\noindent Execution time was measured as wall-clock time (CPU time was not recorded) and was computed from swarm initialization 
to algorithm completion. File operations were not included within this measurement. \newline

\noindent The problem was benchmarked using the $\beta = 2$ case with various swarm sizes and iteration numbers. Unless otherwise 
specified, all data within this thesis was computed as the average of 30 algorithm executions held constant at the given parameters. 
Numerical analysis was additionally performed on the $\beta=4,6,8 \text{ and } 10$ cases, though these solutions were not benchmarked for execution time.

\subsection{MATLAB}
\subsection{Thrust Arcs}
\subsection{Rehyrdration}
MATLAB implementation details go here e.g. ODE45, etc.
\subsection{C++ Single Threaded}
Details about Boost libraries, timers, etc. go here
\subsection{C++ Parallelization}
Details about OpenMP

\newpage