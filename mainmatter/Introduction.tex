%%%%%%%%%%%%%%%%%%%%%%%%%%%%%%%%%%%%%%%%%%%%%%%%
% chapter1.tex                                 %
% Contains formatting and content of chapter 1 %
%%%%%%%%%%%%%%%%%%%%%%%%%%%%%%%%%%%%%%%%%%%%%%%%
\chapter{Introduction and Literature Review}
\newpage


\section{Particle Swarm Optimization}

\noindent Since its introduction in 1995 by Kennedy and Eberhart \citep{Initial_PSO}, particle swarm optimization has had numerous implications in various aerospace fields and beyond.
The algorithm was originally designed to simulate social behavior within a flock of birds but has since been adapted as a stochastic solution for problems
requiring numerical algorithms. Particle swarm optimization consists of a set of possible solutions, referred to as particles, which then explore the solution search space due to
computed velocity update parameters. The velocity change of each particle per algorithm iteration depends on the three things: the particle's best historical position, the best historical
solution within the entire solution set (referred to as the swarm), and the particle's current velocity. In forming the algorithm in this way, many particle swarm optimization techniques 
require no prior knowledge of the solution space and do not necessitate the problem to be differentiable. \newline

\noindent Particle swarm optimization's nature as a metaheuristic algorithm has lent itself to the application of various problems,
including structural design optimization, responsive theater maneuvers, corrosion fatigue, and path-constrained minimal time satellite reorientation
\citep{PSO1, PSO2, PSO3, PSO4}. As a stochastic algorithm, however, a global optimal solution within the search space is not guaranteed for every execution. 
This requires multiple runs of the algorithm and increased computation. As such, significant effort has been devoted to reducing the computational time for this class of
algorithms. One proposed such method is parallel computation. Various papers detail problem-specific parallel particle swarm optimization implementations and the resulting 
effect on computation time \citep{PPSO1, PPSO2, PPSO3}. In addition to algorithm parallelization, this thesis implemented particle swarm optimization within C++.
This development sought to reduce computational
burden with the finer grain memory control synonymous with lower-level programming languages. \newline

\noindent One caveat of particle swarm optimization algorithms is stagnation: a condition in which the swarm converges to a non-optimal solution. The attempt to maintain population 
diversity has been a research area of interest and has seen various approaches in stagnation determination and correction \citep{PSOstag1, PSOstag2}. This thesis introduces a new concept,
dubbed \textit{rehydration}, in which a portion of the population is randomly reset when the population meets the specified stagnation criteria. Analysis of the effect of this
\textit{rehydration} on the search for an optimal solution is explored and presented.

\section{Finite Thrust Transfer}

\noindent Impulsive thrust approximations are used to simplify the computations used in modeling orbital transfers under high-thrust assumptions.
More rigorous analysis of orbital transfers, in correspondence with actual spacecraft maneuvers, requires thrust to be finite. Numerous algorithms have been
proposed to model these finite thrust arcs. Grund and Pitkin proposed an iterative method for computing the optimal thrust arcs requiring the impulsive transfer trajectory as
an initial solution \citep{Fthrust1}. Further literature introduced an indirect solution applying optimal control theory using nonlinear programming from an initially guessed
solution \citep{Fthrust2}. Pontani and Conway then proposed a stochastic algorithm, particle swarm optimization, for the computation of the optimal transfer trajectory \citep{Pontani_Conway}. The problem was formulated as 
a system of 11 unknown parameters and modeled with an objective function aimed at minimizing fuel consumption
and solution error. Consisting of two numerical integrations and requiring lengthy computation time, this finite thrust 
method was thus selected as this thesis' test case for minimizing particle swarm optimization algorithm runtimes.


\section{Thesis Outline}

\noindent Chapter 2 of this thesis defines the generalized particle swarm optimization algorithm and its challenges. This chapter introduces the development of the finite thrust arc problem
and the particle swarm optimization implementation designed to explore its solutions. Additionally, this chapter 
details proposals to reduce both the algorithm runtime and the probability of premature swarm convergence.
Chapter 3 then delves into the problem-specific implementations within MATLAB and C++. Included within
this chapter are details of the parallelized C++ version of the algorithm. 
Chapter 4 then covers the results of these implementations, including speedup comparisons and analysis of the \textit{rehydration} method.
Finally, chapter 5 draws conclusions from these results and offers recommendations for future research. 