%%%%%%%%%%%%%%%%%%%%%%%%%%%%%%%%%%%%%%%%%%%%%%%%
% chapter1.tex                                 %
% Contains formatting and content of chapter 1 %
%%%%%%%%%%%%%%%%%%%%%%%%%%%%%%%%%%%%%%%%%%%%%%%%
\chapter{Introduction and Historical Review}
\newpage

\section{Introduction and Literature Review}

\subsection{Finite Thrust Transfer}

Impulsive thrust approximations are used to simplify the computations used in modeling orbital transfers under high-thrust assumptions.
More rigorous analysis of orbital transfers, in correspondence with actual spacecraft maneuvers, requires thrust to be finite. Numerous algorithms have been
proposed to model these finite thrust arcs. Grund and Pitkin proposed an iterative method for computing the optimal trajectory requiring the impulsive transfer trajectory as
an initial solution \citep{Fthrust1}. Further literature introduces an indirect solution applying optimal control theory using nonlinear programming from an initially guessed
solution \citep{Fthrust2}. Pontani and Conway then proposed a stochastic algorithm, particle swarm optimization, for the computation of the optimal transfer trajectory \citep{Pontani_Conway}. The problem was formulated as 
a system of 11 unknown parameters and modeled with an objective function aimed at minimizing fuel consumption
and solution error. This thesis utilizes this problem formation in its further exploration of the finite thrust transfer problem.

\subsection{Particle Swarm Optimization}

Since its introduction in 1995 by Kennedy and Eberhart \citep{Initial_PSO}, particle swarm optimization has had numerous implications in various aerospace fields and beyond.
The algorithm was originally designed to simulate social behavior within a flock of birds but has since been adapted as a stochastic solution for problems
requiring numerical algorithms. Particle swarm optimization consists of a set of possible solutions, referred to as particles, which then explore the solution search space due to
computed velocity update parameters. The velocity change of each particle per algorithm iteration depends on the three things: the particle's best historical position, the best historical
solution within the entire solution set (referred to as the swarm), and the particle's current velocity. In forming the algorithm in this way, many particle swarm optimization techniques 
require no prior knowledge of the solution space and do not require the problem to be differentiable.

Particle swarm optimization's nature as a metaheuristic algorithm has lent itself to the application of various problems,
including structural design optimization, responsive theater maneuvers, corrosion fatigue, and path-constrained minimal time satellite reorientation
\citep{PSO1, PSO2, PSO3, PSO4}. As a stochastic algorithm, however, a global optimal solution within the search space is not guaranteed for every execution of the algorithm. 
This requires multiple runs of the algorithm and increased computation. As such, significant effort has been devoted to reducing the computational time for this class of
algorithms. One proposed such method parallel computation. Various papers detail problem-specific parallel particle swarm optimization implementations and the resulting 
effect on computation time \citep{PPSO1, PPSO2, PPSO3}. In addition to algorithm parallelization, this thesis proposes a particle swarm optimization implementation within C++.
This development seeks to reduce computational
burden with the finer grain memory control synonymous with lower-level programming languages. 


\section{Thesis Outline}

Chapter 2 of this thesis defines the finite thrust transfer problem to be analyzed. This chapter introduces the development of the problem
and the particle swarm optimization method designed to explore its solutions. Additionally, this chapter details the generalized paralellized particle swarm optimization algorithm.
Chapter 3 then delves into the problem-specific implementations within MATLAB and C++. Included within
this chapter are details on the parallelized C++ version of the algorithm. 
Chapter 4 then details the results of these implementations, including speedup comparisons and methods to greater improve the probability of the algorithm approaching an optimal solution.
Finally, chapter 5 explores draws conclusions from these results and offers recommendations for future research. 