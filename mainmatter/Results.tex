%%%%%%%%%%%%%%%%%%%%%%%%%%%%%%%%%%%%%%%%%%%%%%%%
% chapter4.tex                                 %
% Contains formatting and content of chapter 4 %
%%%%%%%%%%%%%%%%%%%%%%%%%%%%%%%%%%%%%%%%%%%%%%%%
\chapter{Results}


\section{Numerical Results}

\noindent Implementation of the particle swarm optimization algorithm provides a way to solve
for the optimal orbit transfer between two circular orbits. Table (\ref{tab:Numerical-Results}) shows
the results of these computations. The ratio $(\frac{m_f}{m_{0}})_h$ represents the final to initial mass
ratio for a Hohmann transfer between orbits, whereas $\frac{m_f}{m_0}$ is the mass ratio of the algorithm 
determined optimal solution.


\begin{table}[H]
  \centering
  \begin{tabular}{@{}lllllll@{}}
  \toprule
  $\beta$ & $\Delta t_1$ & $\Delta t_{co}$ & $\Delta t_2$ & $J$ & $m_f/m_0$ & $(m_f/m_0)_h$ \\ \midrule
  2 & 0.671 & 5.229 & 0.411 & 1.082 & 0.567062 & 0.56614 \\
  4 & 1.044 & 11.834 & 0.442 & 1.487 & 0.405365 & 0.407642 \\
  6 & 1.178 & 20.006 & 0.413 & 1.59 & 0.3638 & 0.368367 \\
  8 & 1.25 & 24.87 & 0.403 & 1.652 & 0.339164 & 0.353299 \\
  10 & 1.282 & 41.08 & 0.365 & 1.647 & 0.34106 & 0.346603 \\ \bottomrule
  \end{tabular}
  \caption{Numerical results for the optimum orbital transfer for different $\beta$ values }
  \label{tab:Numerical-Results}
  \end{table}

\noindent For all values of $\beta$ in Table (\ref{tab:Numerical-Results}) the computed final to initial mass ratio is similar to the final to 
initial mass ratio for a Hohmann transfer. For $\beta = 2$, this value exceeds its Hohmann
counterpart. This indicates that this algorithm is effective in determining optimal orbit trajectories without
requiring the impulsive thrust assumption as made in the Hohmann transfer calculations.

\begin{figure}[H]
    \includegraphics[width=\linewidth]{./jpgs/thrustArcB2.jpg}
    \caption{Numerically computed optimal transfer trajectory, $\beta = 2$.}
    \label{fig:Tarc-B2}
  \end{figure}

\noindent Fig. (\ref{fig:Tarc-B2}) displays the numerically computed optimal transfer
trajectory for $\beta = 2$. This yields insight into the optimal shape of the transfer orbit; a low 
eccentricity, relatively short orbit path connects the $R_2 = 2 LU$ and $R_1 = 1 LU$ circular orbits.

  \begin{figure}[H]
    \includegraphics[width=\linewidth]{./jpgs/thrustArcB10.jpg}
    \caption{Numerically computed optimal transfer trajectory, $\beta = 10$.}
    \label{fig:Tarc-B10}
  \end{figure}

\noindent Analyzing Fig. (\ref{fig:Tarc-B10}) shows how the transfer orbit trajectory differs between two orbits 
with higher $\beta = \frac{R_2}{R_1}$ values. In comparison to Fig. (\ref{fig:Tarc-B2}), the transfer trajectory has
a much higher eccentricity and longer duration. In can thus be graphically and logically inferred that transfer orbits between orbits 
with increasingly larger $\beta$ values are increasingly elliptical.

\begin{figure}[H]
\includegraphics[width=\linewidth]{./jpgs/thrustAnglesB2.jpg}
\caption{Numerically computed thrust-angle time histories for optimal $\beta$ = 2 solutions  }
\label{fig:thrustAnglesB2}
\end{figure}

\noindent Particle swarm optimization's stochastic nature does not guarantee optimal results every 
run, nor does it guarantee that particles with similar cost function values represent similar solutions. 
Fig. (\ref{fig:thrustAnglesB2}) shows the thrust-pointing-angle time histories for both thrust arcs for 
different particles each with quasi-optimal cost function values. While similar in cost function value, the 
thrust-pointing-angle varies greatly for these three particles. Thus, it can be observed that within a search 
space of 11 unknowns, near optimum solutions may vary greatly in their characteristics.

\begin{figure}[H]
\includegraphics[width=\linewidth, height=13.5cm]{./jpgs/i50.jpg}
\caption{Global Best J Value Per Iteration: 100 Particles over 50 iterations}
\label{fig:gBestPer50Iter}
\end{figure}

\noindent Fig. \ref{fig:gBestPer50Iter} demonstrates the evolution of the best solution
found throughout the first 50 iterations of the algorithm. Two key results can be ascertained 
through this plot: First, the lowest found cost function value decreases rapidly over the first few
iterations. Second, the swarm may appear to stagnate at a cost function value then make a jump to a
better solution. This notion makes it difficult to discern when a population has truly stagnated. 

\begin{figure}[H]
    \includegraphics[width=\linewidth, height=13.5cm]{./jpgs/i500.jpg}
    \caption{Global Best J Value Per Iteration: 110 Particles over 500 iterations}
    \label{fig:gBestPer500Iter}
    \end{figure}

\noindent Fig. ()\ref{fig:gBestPer500Iter}) showcases the evolution of global best cost function value over a larger 
range of iterations. This figure reinforces the observations from fig. (\ref{fig:gBestPer500Iter}). The rapid decrease 
in cost function value is generally limited to the beginning iterations, and cost function value
plateaus in the later stages of the algorithm. One notable portion is the significant decrease in the global best 
value for run four following over 300 iterations. This event demonstrates the inherent randomness of the particle swarm 
optimization algorithm, and once again illustrates how population stagnation can be difficult to predict.

\section{Rehydration Results}

\noindent \textit{Rehydration} is a method proposed within this thesis that resets a portion of the swarm if the population is estimated to 
be stagnated. This method is designed to inject more randomness into the population to explore more of the solution search space, While
simultaneously exploiting previously computed knowledge of the solution set. 

\noindent Results within this section are derived from the variation of the three main rehydration parameters: $\delta_{Jcrit}$, $\eta_{iter}$,
and $P_{r,\text{\%}}$. 

\begin{table}[H]
  \centering
  \begin{tabular}{lll|ll|ll}
    \toprule
    \multirow{2}{*}{$P_{r,\text{\%}}$ = 25} &
      \multicolumn{2}{c}{$\delta_{Jcrit} = .1\%$  } &
      \multicolumn{2}{c}{$\delta_{Jcrit} =  1\%$} &
      \multicolumn{2}{c}{$\delta_{Jcrit} = 5\%$} \\
      \cmidrule{2-7}
    & $\bar{J}_{best}$ & $\Delta \bar{J}_{best,\text{\%}}$ & $\bar{J}_{best}$ & $\Delta \bar{J}_{best,\text{\%}}$ & $\bar{J}_{best}$ & $\Delta \bar{J}_{best,\text{\%}}$ \\
    \midrule

    $n_{iter}=5$ & 1.347 & 44.177\% & 1.54 & 36.179\% & 1.438 & 40.406\% \\
    $n_{iter}=10$ & 1.421 & 41.11\% & 1.379 & 42.851\% & 1.594 & 33.941\% \\
    $n_{iter}=20$ & 1.331 & 44.840\% & 1.652 & 31.538\% & 1.503 & 37.712\% \\
    \bottomrule
  \end{tabular}
  \caption{Rehydration results for a 25\% population reset}
  \label{tab:rehydation-p25}
\end{table}

\begin{table}[H]
  \centering
  \begin{tabular}{lll|ll|ll}
    \toprule
    \multirow{2}{*}{$P_{r,\text{\%}}$ = 33} &
      \multicolumn{2}{c}{$\delta_{Jcrit} = .1\%$ } &
      \multicolumn{2}{c}{$\delta_{Jcrit} = 1\%$ } &
      \multicolumn{2}{c}{$\delta_{Jcrit} = 5\%$ } \\
      \cmidrule{2-7}
    & $\bar{J}_{best}$ & $\Delta \bar{J}_{best,\text{\%}}$ & $\bar{J}_{best}$ & $\Delta \bar{J}_{best,\text{\%}}$ & $\bar{J}_{best}$ & $\Delta \bar{J}_{best,\text{\%}}$ \\
    \midrule

    $n_{iter}=5$ & 1.371 & 43.18\% & 1.475 & 38.872\% & 1.412 & 41.484\% \\
    $n_{iter}=10$ & 1.364 & 43.47\% & 1.558 & 35.433\% & 1.473 & 40.448\% \\
    $n_{iter}=20$ & 1.412 & 41.48\% & 1.531 & 36.552\% & 1.578 & 34.604\% \\
    \bottomrule
  \end{tabular}
  \caption{Rehydration results for a 33\% population reset}
  \label{tab:rehydation-p33}
\end{table}


\begin{table}[H]
    \centering
    \begin{tabular}{lll|ll|ll}
      \toprule
      \multirow{2}{*}{$P_{r,\text{\%}} = 50$} &
        \multicolumn{2}{c}{$\delta_{Jcrit} = .1\%$ } &
        \multicolumn{2}{c}{$\delta_{Jcrit} = 1\%$ } &
        \multicolumn{2}{c}{$\delta_{Jcrit} = 5\%$ } \\
        \cmidrule{2-7}
      & $\bar{J}_{best}$ & $\Delta \bar{J}_{best,\text{\%}}$ & $\bar{J}_{best}$ & $\Delta \bar{J}_{best,\text{\%}}$ & $\bar{J}_{best}$ & $\Delta \bar{J}_{best,\text{\%}}$ \\
      \midrule

      $n_{iter}=5$ & 1.58 & 34.52\% & 1.469 & 39.12\% & 1.425 & 40.94\% \\
      $n_{iter}=10$ & 1.396 & 42.15\% & 1.306 & 45.88\% & 1.473 & 38.96\% \\
      $n_{iter}=20$ & 1.373 & 43.10\% & 1.39 & 42.40\% & 1.33 & 44.88\% \\
      \bottomrule
    \end{tabular}
    \caption{Rehydration results for a 50\% population reset}
    \label{tab:rehydation-p50}
  \end{table}

\section{Speedup}

\subsection{MATLAB}

%Putting the H here ensures the table goes exactly where you put
%it in the code
\begin{table}[H]
    \centering
    \begin{tabular*}{.5\textwidth}{c @{\extracolsep{\fill}} cc}
    \toprule
    \bm{$P_{num}$} & \bm{$I_{num}$} & \textbf{Time (s)} \\ \midrule
    25                        & 250            & 30.527   \\
    25                        & 500            & 39.15    \\
    25                        & 1000           & 68.88    \\
    50                        & 250            & 30.527   \\
    50                        & 500            & 80.4246  \\
    50                        & 1000           & 141.151  \\
    100                      & 250            & 95.956   \\
    100                      & 500            & 152.22   \\
    100                      & 1000           & 306.732  \\
    150                      & 250            & 161.3    \\
    150                      & 500            & 195.87   \\
    150                      & 1000           & 388.03   \\
    200                      & 250            & 157.061  \\
    200                      & 500            & 251.575  \\
    200                      & 1000           & 457.55   \\ \bottomrule
    \end{tabular*}
    \caption{MATLAB Numerical Results - Wall Clock Time}
    \label{tab:MATLAB-speedup}
    \end{table}

\subsection{C++ Single Threaded}

% Please add the following required packages to your document preamble:
% \usepackage{booktabs}
\begin{table}[H]
    \centering
    \begin{tabular}{@{}lllll@{}}
    \toprule
    \bm{$P_{num}$} & \bm{$I_{num}$} & \textbf{Time (s)} & \bm{$n_{su\%}$} & \bm{$n_{su}$} \\ \midrule
    25          & 250  & 1.732    & 94.32633406          & 17.62528868              \\
    25          & 500 & 2.39     & 93.89527458          & 16.38075314              \\
    25          & 1000 & 3.92     & 94.30894309          & 17.57142857              \\
    50          & 250 & 2.96     & 90.30366561          & 10.31317568              \\
    50          & 500 & 3.26     & 95.94651388          & 24.6701227               \\
    50          & 1000 & 5.695    & 95.96531374          & 24.78507463              \\
    100         & 250 & 6.659    & 93.060361            & 14.40997147              \\
    100         & 500 & 7.21     & 95.2634345           & 21.11234397              \\
    100         & 1000 & 11.83    & 96.14321297          & 25.92831784              \\
    150         & 250 & 7.26     & 95.49907006          & 22.21763085              \\
    150         & 500 & 12.64    & 93.54674018          & 15.4960443               \\
    150         & 1000 & 16.68    & 95.7013633           & 23.26318945              \\
    200         & 250 & 7.77     & 95.05287754          & 20.21377091              \\
    200         & 500 & 11.423   & 95.45940574          & 22.02354898              \\
    200         & 1000 & 21.06    & 95.39722435          & 21.72602089              \\ \bottomrule
    \end{tabular}
    \caption{C++ Single Threaded Speedup Results - Wall Clock Time}
    \label{tab:ST-speedup}
    \end{table}


\subsection{C++ OpenMP}
\begin{table}[H]
    \centering
    \begin{tabular}{@{}llllll@{}}
    \toprule
    \bm{$P_{num}$} & \bm{$I_{num}$} & \textbf{Time (s)} & \bm{$n_{su\%}$} & \bm{$n_{sust\%}$} & \bm{$n_{su}$} \\ \midrule
    25  & 250 & 1.78  & 94.16909621 & -2.771362587 & 17.15       \\
    25  & 500& 3.4   & 91.31545338 & -42.25941423 & 11.51470588 \\
    25  & 1000& 5.23  & 92.40708479 & -33.41836735 & 13.17017208 \\
    50  & 250 & 2.81  & 90.7950339  & 5.067567568  & 10.86370107 \\
    50  & 500& 4.77  & 94.06897889 & -46.3190184  & 16.86050314 \\
    50  & 1000& 8.51  & 93.9709956  & -49.42932397 & 16.58648649 \\
    100 & 250& 5.71  & 94.04935595 & 14.2513891   & 16.80490368 \\
    100 & 500& 6.6   & 95.66417028 & 8.460471567  & 23.06363636 \\
    100 & 100 & 11.73 & 96.17581472 & 0.8453085376 & 26.14936061 \\
    150 & 250 & 4.92  & 96.94978301 & 32.23140496  & 32.78455285 \\
    150 & 500 & 9.77  & 95.01199775 & 22.7056962   & 20.04810645 \\
    150 & 1000& 13.93 & 96.41007139 & 16.48681055  & 27.85570711 \\
    200 & 250 & 5.88  & 96.25623166 & 24.32432432  & 26.71105442 \\
    200 & 500 & 10.46 & 95.84219418 & 8.4303598    & 24.05114723 \\
    200 & 1000 & 22.42 & 95.09998907 & -6.457739791 & 20.40811775 \\ \bottomrule
    \end{tabular}
    \caption{OpenMP Speedup Results - Wall Clock Execution Time}
    \label{tab:OpenMP-speedup}
    \end{table}

\newpage