%%%%%%%%%%%%%%%%%%%%%%%%%%%%%%%%%%%%%%%%%%%%%%%%
% chapter2.tex                                 %
% Contains formatting and content of chapter 2 %
%%%%%%%%%%%%%%%%%%%%%%%%%%%%%%%%%%%%%%%%%%%%%%%%
\chapter{Problem Statement}

\section{Problem Definition}

This thesis analyzes the optimal finite thrust transfer between two circular orbits. A PSO approach is utlized to attempt
to find and optimal solution. The development of this problem is derived from \citep{Pontani_Conway}. \newline

This transfer is developed with respect to an initial circular orbit of radius $R_1$ and a final circular orbit of radius $R_2$ subject to the conditions $R_1 > R_2$ where the parameter $\beta = R_2/R_1 > 1$. 
The inertial reference frame selected for this problem definition is centered at the attracting body. The corresponding coordinate frame definition is as follows: The $x$ axis is aligned with the spacecraft at the initial time $t_0$ and the $y$ axis is located in the orbital plane.
The $z$ axis is determined by the right hand rule of these two basis axes. Within this problem $v_r$ denotes the spacecraft radial velocity, $v_\theta$ the horizontal component of velocity, $r$ the radius, and $\zeta$ the angular displacement from the $x$ axis. The gravitational
parameter of the attracting body is $\mu_B$. \newline

\noindent Initial conditions at time $t_0$ are given by
\begin{equation}
v_r(t_0) = 0  \quad  v_\theta(t_0) = \sqrt{\mu_B/R_1} \quad r(t_0) = R_1 \quad \xi(t_0) = 0
\label{initial_conditions}
\end{equation}

\noindent And final conditions given by
\begin{equation}
    v_r(t_f) = 0 \quad v_\theta(t_f) = \sqrt{\mu_B/R_2} \quad r(t_f) = R_2
    \label{final_conditions}
\end{equation}

The problem assumes a transfer trajectory of an initial thrust arc, followed by a coasting arc, and ending with a second thrust arc. 
Optimization of this problem corresponds to determining the thrust pointing angle function corresponding to the minimization of propellant consumption 
subject to the constraints of Eq. (\ref{final_conditions}).\newline

\noindent Additional assumptions are made to simplify the analysis:

\begin{enumerate}
    \item Throughout the duration of each thrust arc maximum thrust is generated
    \item Each thrust arc pointing angle is represented as a third-degree polynomial as a function of time.
\end{enumerate}

$T$ and $c$ are used to represent the spacecraft thrust level and effective exhaust velocity respectively. The previous assumptions
indicate that the thrust-to-mass ratio has the form

\begin{equation}
\dfrac{T}{m} = \begin{cases} 
    \dfrac{T}{m_0-\frac{T}{c}t} = \dfrac{cn_0}{c-n_0t} & 0\leq t \leq t_1 \\
    0 & t_1\leq t \leq t_2 \\
    \dfrac{T}{m_0-\frac{T}{c}(t_1+t-t_2)} = \dfrac{cn_0}{c-n_0(t_1+t-t_2)} & t_2\leq t \leq t_f 
  \end{cases}
  \label{Toverm}
\end{equation}

\noindent $m_0$ is the initial spacecraft mass and $n_0$ is the initial thrust to mass ratio at $t_0$.
The state space equations of motion for the spacecraft are

\begin{equation}
    \dot{v} = -\dfrac{\mu_B-rv_\theta^2}{r^2}+\dfrac{T}{m}\sin\delta
    \label{vrdot_eom}
\end{equation}
\begin{equation}
\dot{v_\theta} = -\dfrac{v_rv_\theta}{r}+\dfrac{T}{m}\cos\delta
\label{vthetadot_eom}
\end{equation}
\begin{equation}
    \dot{r} = v_r
    \label{rdot_eom}
\end{equation}
\begin{equation}
    \label{xidot_eom}
\dot{\xi} = \dfrac{v_\theta}{r}
\end{equation}
where $\dfrac{T}{m}$ is given in E.q.(\ref{Toverm}) and $\delta$ is the thrust pointing angle represent by a
third-order polynomial of time. The state vector used in this problem is $[x_1 \; x_2 \; x_3 \; x_4]^T = [ v_r \; v_\theta \; r \; \xi ]^T$. \linebreak

\noindent The thurst pointing angle $\delta$ is defined as 

\begin{equation}
    \delta = 
    \begin{cases}
        \zeta_0 +\zeta_1t+\zeta_2t^2 +\zeta_3t^3 & 0 \leq t \leq t_1 \\
        \nu_0 + \nu_1(t-t_2) + \nu_2(t-t_2)^2+\nu_3(t-t_3)^3 & t_2 \leq t \leq t_f
    \end{cases}
    \label{delta_eq}
\end{equation} \linebreak

\noindent The optimum thrust pointing angle coefficients $\{\zeta_0 \; \zeta_1 \; \zeta_2 \; \zeta_3 \}$
and $\{\nu_0 \; \nu_1 \; \nu_2 \; \nu_3\}$ are determined by PSO. \newline

During the coasting arc the problem consists of a Keplerian orbit. As such, the semimajor axis $a$ and eccentricity $e$ of the coasting arc can be computed as

\begin{equation}
a = \dfrac{\mu_Br_1}{2\mu_B - r_1(v_{r1}^2+v_{\theta1}^2)}
\label{acoast}
\end{equation}

\begin{equation}
e = \sqrt{1-\dfrac{r_1^2v_{\theta_1}^2}{\mu_Ba}}
\label{ecoast}
\end{equation} \newline

\noindent Provided that the orbit is elliptic ($a > 0$) the true anomaly at $t_1(f_1)$ can be computed as

\begin{equation}
\sin f_1 = \dfrac{v_{r1}}{e}\sqrt{\dfrac{a(1-e^2)}{\mu_B}} \quad \text{and} \quad \cos{f_1} = \dfrac{v_{\theta_1}}{e}\sqrt{\dfrac{a(1-e^2)}{\mu_B}}-\dfrac{1}{e}
\label{trueAnamolyCoast}
\end{equation} \newline

\noindent and the eccentric anomaly $E_1$ as 

\begin{equation}
\sin{E_1} = \dfrac{\sin{f_1}\sqrt{1-e^2}}{1+e\cos{f_1}} \quad \text{and} \quad 
\cos{E_1} = \dfrac{\cos{f_1}+e}{1+e\cos{f_1}}
\label{eccAnamolyCoast}
\end{equation}

\noindent The PSO parameter $\Delta E$ represents the variation in eccentric anomaly througout the coasting arc. Therefore, the eccentric enomaly at $t_2$ is $E_2 = E_1 + \Delta E$. True anamoly can be
thus determined utilizing the results of Eq. (\ref{eccAnamolyCoast})

\begin{equation}
\sin{f_2} = \dfrac{\sin{E_2}\sqrt{1-e^2}}{1-e\cos{E_2}} \quad \text{and} \quad
\cos{f_2} = \dfrac{\cos{E_2}-e}{1-e\cos{E_2}}
\label{trueanamoly2}
\end{equation}

\noindent Furthermore, the coasting time interval $t_{co}$ can be calculated through Kepler's law as
\begin{equation}
t_{co} \overset{\Delta}{=} t_2 - t_1 = \sqrt{\dfrac{a^3}{\mu_B}}[E_2-E_1-e(\sin{E_2}-\sin{E_1})]
\end{equation}

\noindent The results from Eqs. (\ref{acoast}, \ref{ecoast}, and \ref{trueanamoly2}) provide the initial conditions required to numerical integrate 
the spacecraft equations of motion for the second thrust arc beginning at time $t_2$. These initial conditions for the second thrust arc are computed as

\begin{equation}
v_{r_2} = \sqrt{\dfrac{\mu_B}{a(1-e^2)}}e\sin{f_2}
\label{vr2}
\end{equation}

\begin{equation}
    v_{\theta2} = \sqrt{\dfrac{\mu_B}{a(1-e^2)}}(1+e\cos{f_2})
    \label{vtheta2}
\end{equation}

\begin{equation}
r_2 = \dfrac{a(1-e^2)}{1+e\cos{f_2}}
\label{r2}
\end{equation}

\begin{equation}
\xi_2 = \xi_1 + (f_2-f_1)
\end{equation}

\noindent Integrating the second thrust arc with these initial conditions over the second thrust arc time duration with yield the final orbital characteristics. \newline

This system depends on the eight coefficients representing the thrust pointing angles of the first and second thrust arcs, 
$\{\zeta_0 \; \zeta_1 \; \zeta_2 \; \zeta_3 \}$ and $\{\nu_0 \; \nu_1 \; \nu_2 \; \nu_3 \}$ respectively, and the time intervals for the first coast arc
$\Delta t_1 \overset{\Delta}{=} t_1$, the coasting arc, $\Delta t_{co}$, and the second coast arc $\Delta t_2 \overset{\Delta}{=} t_f-t_2 $. These 11 unknowns are sought to be chosen to minimize the objective function 

\begin{equation}
J = \Delta t_1 + \Delta t_2
\label{J}
\end{equation}

\noindent Optimizing the unknown coefficients to minimize \ref{J} corresponds to the minimization of propellant consumption. Minimizing propellant consumption
results in a maximization of the final-to-initial mass ratio given by 

\begin{equation}
\dfrac{m_f}{m_0} = \dfrac{m_0-\frac{T}{c}\Delta t_1-\frac{T}{c}\Delta t_2}{m_0} = 1-\dfrac{n_0}{c}(\Delta t_1 +\Delta t_2)
\label{finalToInitialMassRatio}
\end{equation}

\noindent The coasting arc is required to be elliptic, therefore any particles with $a \leq 0$ are assigned an infinite value. \newline

Since $\Delta t_{co}$ can be computed using $\Delta E$, the eccentric anomaly variation replaces the coasting time interval in the 
problem's 11 unkown parameters. Each particle in the swarm thus consists of the following

\begin{equation}
\chi = [ \zeta_0 \quad \zeta_1 \quad \zeta_2 \quad \zeta_3 \quad \nu_0 \quad \nu_1 \quad \nu_2 \quad \nu_3 \quad \Delta t_1 \quad \Delta E \quad \Delta t_2 ]^T
\label{particleUnkowns}
\end{equation}

To enforce the constraints set forth in Eq. (\ref{final_conditions}) three penalty terms are added to the objective function Eq. (\ref{J}).

\begin{equation}
\tilde{J} = \Delta t_1 + \Delta t_2 + \sum_{k=1}^3 \alpha_k|d_k| 
\label{JwithPenalty}
\end{equation}

\noindent where 

\begin{equation}
d_1 = v_r(t_f) \quad \quad d_2 = v_\theta(t_f) - \sqrt{\dfrac{\mu_B}{R_2}} \quad \quad 
d_3 = r(t_f)-R_2
\label{penaltyValues}
\end{equation}

\noindent A maximum acceptable error of $10^{-3}$ is used and the $\alpha_k$ coefficients assigned as  

\begin{equation}
    \label{penaltyCoefficients}
\alpha_k = \begin{cases}
    100 & |d_k| > 10^{-3} \\
    0 & |d_k| < 10^{-3}
\end{cases}
\end{equation}


\section{Canonical Units Definition}

This problem employs a canonical set of units to simplify the analysis. One distance unit (DU) is defined as the radius of the initial orbit,
and one time unit (TU) definted such that $\mu_B = 1 \frac{DU^3}{TU^2}$. The unknown coefficients are thus sought within the following ranges

\begin{equation}
\label{particleBounds}
\begin{gathered}
0 \; TU \leq t_1 \leq 3 \; TU \quad 0 \leq \Delta E \leq 2\pi \quad 0 \; TU \leq \Delta t_2 \leq 3 \; TU \\
-1 \leq \xi_k \leq 1 \quad -1 \leq \nu_k \leq 1 \quad (k = 0,1,2,3)
\end{gathered}
\end{equation}

\noindent The effective exhaust velocity $c$ is set to $0.5 \frac{DU}{TU^2}$ and the initial thrust-to-mass ratio set to $0.2 \frac{DU}{TU^2}$. 

\section{PSO Algorithm}

\subsection{General Overview}
Particle Swarm Optimization is a class of algorithm that employs statistical methods to attempt to locate optimal solutions 
that minimize an objective function, often denoted $J$, in the global search space. The algorithm was originally designed to model
the behavior of flocks of birds or schools of fish but has been adapted to solve a variety of mathematical problems. PSO is an effective method in scenarios where no analytical solution can be found and the optimal solution is unknown. \newline

The potential solutions for PSO algorithms exist within an $n$ dimensional space where $n$ represents the number of unknown parameter values. Within this $n$ dimensional space there is likely to 
be a variety of local minima in which the swarm, and thus solution, may converge on. As such, one is not guaranteed to find an optimal solution to a problem with a PSO algorithm. Indeed, in many
cases it is impossible to know what the true global optimum is. This requires running the algorithm many times to increase the probability of finding a near-optimum solution. 
For many problems, this may require large computational efforts and extensive time. Reducing the time required for the execution of PSO algorithms allows more runs in a shorter duration,
thus probabilistic increasing the chance of a near-optimal solution quicker. This opportunity was one of the core concepts explored within this thesis.  \newline

The algorithm consists of a swarm of particles, each initially containing randomly assigned values in the search space for each unknown
parameter. These parameter values are referred to as a particles position, with each parameter also having a corresponding velocity. A given number of iterations is set, each in which
each particle is evaluated for its fitness as defined by the objective function $J$. Following each iteration, each of the particles' parameters' position is updated with its velocity terms,
and velocity updated based on a variety of factors
\begin{enumerate}
    \item How far each of the particle's parameters differs from the global best particle ever recorded's parameters
    \item How far each of the particle's parameters differs from the parameters of its historical personal best values
    \item Its current velocity, where velocity refers to the rate of change of a particle's position in the search space per iteration
\end{enumerate}

\noindent This process continues for the set number of iterations. The global best particle, i.e. set of unknown parameters, found 
within the PSO algorithm is thus the solution. Psuedocode for the algorithm is as follows

\begin{algorithm}[H]
    \caption{General PSO Algorithm}
    \begin{algorithmic}
    
    \STATE Set Lower and Upper bounds on position and velocity
    \FOR{ \text{\emph{particle }} \text{ \textbf{in} \emph{Swarm}}}
    \STATE Assign initial position values for each particle
    \ENDFOR
    \FOR{ \text{\emph{iteration}} \text{ \textbf{in} \emph{Num Iterations}}}
    \FOR {\text{\emph{particle}} \text{ \textbf{in}  \emph{Swarm}}}
    \STATE Evaluate objective function $J$
    \ENDFOR
    \STATE Record best set of parameters each particle has achieved $\leftarrow pBest$
    \STATE Record best overall position as global best $\leftarrow gBest$
    \FOR {\text{\emph{particle}} \text{ \textbf{in}  \emph{Swarm}}}
    \STATE Update particle velocity based on $pBest, gBest$, and current velocity
    \IF{$v_i(k) < $ velocity lower bound}
    \STATE $v_i(k) = $ velocity lower bound
    \ELSIF{$v_i(k) > $ velocity upper bound}
    \STATE $v_i(k) = $ velocity upper bound
    \ENDIF
    \STATE Update particle Position
    \IF{$p_i(k) < $ position lower bound}
    \STATE $p_i(k) = $ position lower bound
    \STATE $v_i(k) = 0$
    \ELSIF{$p_i(k) > $ position upper bound}
    \STATE $p_i(k) = $ position upper bound
    \STATE $v_i(k) = 0$
    \ENDIF
    \ENDFOR
    \ENDFOR
    \STATE Use $gBest$ as solution
    \end{algorithmic}

\end{algorithm}

\subsection{Finite Thrust Arc Implementation}

The Finite Thrust Arc problem explored within this thesis was implemented using the 11 unknown parameters in equation Eq. (\ref{particleUnkowns})
and corresponding objective function Eq. (\ref{JwithPenalty}). Position and velocity bounds for the PSO algorithm are given by Eq. (\ref{particleBounds}). Particle position updating uses 
the standard form

\begin{equation}
    \label{positionUpdating}
p_i = p_i+v_i
\end{equation}

\noindent $i$ ranges from $1$ to the $11$ unknowns. \newline

\noindent Velocity updating is implemented as

\begin{equation}
    \label{eq:velocityUpdating}
v_i = v_ic_i+c_c(pBest_i-p_i)+c_s(gBest-Pi)
\end{equation}

\noindent where the three accelerator coefficients are defined as

\begin{equation}
    \label{eq:acceleratorCoefficients}
c_i = \dfrac{(1+rand(0,1))}{2} \quad c_c = 1.49445(rand(0,1)) \quad c_s = 1.49445(rand(0,1))
\end{equation}

\noindent and $rand(0,1)$ is a uniformly distributed random number generated between 0 and 1. \newline

\noindent A detailed look at the main portion of the PSO psuedocode is shown below

\begin{algorithm}[H]
    \caption{Main Finite Thrust Transfer PSO Algorithm}
    \begin{algorithmic}

    \FOR{ \text{\emph{iteration}} \text{ \textbf{in} \emph{Num Iterations}}}
    \FOR {\text{\emph{particle}} \text{ \textbf{in}  \emph{Swarm}}}
    \STATE Numerically integrate first thrust arc using state-space equations \ref{vrdot_eom},
    \ref{vthetadot_eom}, \ref{rdot_eom}, and \ref{xidot_eom}
    \STATE Compute initial conditions for second thrust arc using equations \ref{acoast}, \ref{ecoast}, and \ref{trueanamoly2}
    \STATE Numerical integrate second thrust arc using state-space equations
    \STATE Evaluate $\tilde{J}$ (\ref{JwithPenalty}) numerically integrated second thrust arc values and penalty terms \ref{penaltyValues} and
    penalty coefficients \ref{penaltyCoefficients}
    \ENDFOR
    \FOR {\text{\emph{particle}} \text{ \textbf{in}  \emph{Swarm}}}
    \STATE Update particle velocity with equation \ref{velocityUpdating}
    \STATE Check velocity bounds
    \STATE Update particle Position with equation \ref{positionUpdating}
    \STATE Check position bounds
    \ENDFOR
    \ENDFOR
    \STATE Use $gBest$ as solution
    \end{algorithmic}

\end{algorithm}

\section{Reducing Computational Time}

\subsection{Problem Selection} \label{problemSelection}
The finite thrust arc problem explored within this thesis was chosen as a benchmark as a PSO algorithm requiring 
lots of computational resources. Particles within this problem require lots of execution time for two main reasons

\begin{enumerate}
    \item Each particle requires two numerical integrations, one for the initial thrust arc and one for the second
    \item Many potential solutions within the global search space do not converge during these numerical integrations. This drastically reduces integrations
    integration step sizes in an attempt to meet the specified integration tolerances, massively increasing the computational complexity. 
\end{enumerate}

Additionally, this problem required a non-trivial 11 unknowns. This broad 11 dimensional search space necessitates a comparatively large swarm population size
in the search for optimal values. This combination of individual particles requiring lots of execution with the desired swarm population size makes the problem explored within this thesis
was sufficient in evaluating potential speedup methods.

\subsection{Parallelization}

The conditions referenced within section \ref{problemSelection} make the finite thrust transfer problem a good candidate for parallelization.
Within each iteration, each particle is evaluated for its value of the objective function $J$. This  contains the computationally heavy numerical integrations. Within this section of the PSO algorithm, each particle is evaluated independently. 
As such, parallel processing can be used to evaluate multiple particles within the swarm simultaneously. Only after each iteration is complete and position and velocity updating occurs do the particles become dependent on each others' results. Since 
the updating portion of the algorithm is not computationally expensive, a large potential benefit can be gained from paralellizing the inter-iteration computations. A simplified version of the parallelized algorithm is shown below.

\begin{algorithm}[H]
    \caption{Simplified parallel PSO Implementation}
    \begin{algorithmic}
    \FOR{ \text{\emph{iteration}} \text{ \textbf{in} \emph{Num Iterations}}}
    \FOR {\text{\emph{particle}} \text{ \textbf{in}  \emph{Swarm}}}
    \STATE Evaluate objective function $J$
    \hspace{17em}\smash{$\left.\rule{0pt}{1.5\baselineskip}\right\}\ \mbox{in parallel}$}

    \ENDFOR
    \STATE Update particle velocity based on $pBest, gBest$, and current velocity
    \STATE Update particle position
    \ENDFOR
    \end{algorithmic}
\end{algorithm}

\newpage
