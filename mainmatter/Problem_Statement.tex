%%%%%%%%%%%%%%%%%%%%%%%%%%%%%%%%%%%%%%%%%%%%%%%%
% chapter2.tex                                 %
% Contains formatting and content of chapter 2 %
%%%%%%%%%%%%%%%%%%%%%%%%%%%%%%%%%%%%%%%%%%%%%%%%
\chapter{Problem Statement}

\section{Particle Swarm Optimization Overview}

\subsection{Algorithm Definition}
\noindent Particle swarm optimization is a class of algorithm that employs statistical methods to locate optimal solutions 
minimizing an objective function in the global search space. The algorithm was originally designed to model
the behavior of bird flocks but has since been adapted to solve a variety of mathematical problems.
Inherently metaheuristic, particle swarm optimization requires no
prior knowledge of the search space. \newline

\noindent Potential solutions for particle swarm optimization methods exist within an $n$-dimensional unknown parameter space.
The algorithm consists of a population, or swarm, of particles initially seeded with randomly assigned parameter sets.
A particle maintains both its position - an $n$-dimensional parameter array - and a corresponding $n$-dimensional velocity vector.
For each iteration, the fitness of every particle's candidate solution is evaluated as defined by the objective function $J$. 
Following the iteration, every particles' position is updated with its velocity terms,
and velocity vector updated according to
\begin{enumerate}
    \item How far each of the particle's parameters differs from the global best particle ever recorded's parameters
    \item How far each of the particle's parameters differs from the parameters of its historical personal best values
    \item Its current velocity, where velocity refers to the rate of change of a particle's position in the search space per iteration
\end{enumerate}

\noindent This process continues for the set number of iterations. The global best particle at algorithm completion
is the discovered solution. Pseudocode for the algorithm is as follows

\begin{algorithm}[H]
    \caption{General PSO Algorithm}
    \begin{algorithmic}
    
    \STATE Set Lower and Upper bounds on position and velocity
    \FOR{ \text{\emph{particle }} \text{ \textbf{in} \emph{Swarm}}}
    \STATE Assign initial position values for each particle
    \ENDFOR
    \FOR{ \text{\emph{iteration}} \text{ \textbf{in} \emph{Num Iterations}}}
    \FOR {\text{\emph{particle}} \text{ \textbf{in}  \emph{Swarm}}}
    \STATE Evaluate objective function $J$
    \ENDFOR
    \STATE Record best set of parameters each particle has achieved $\leftarrow pBest$
    \STATE Record best overall position as global best $\leftarrow gBest$
    \FOR {\text{\emph{particle}} \text{ \textbf{in}  \emph{Swarm}}}
    \STATE Update particle velocity based on $pBest, gBest$, and current velocity
    \IF{$v_i(k) < $ velocity lower bound}
    \STATE $v_i(k) = $ velocity lower bound
    \ELSIF{$v_i(k) > $ velocity upper bound}
    \STATE $v_i(k) = $ velocity upper bound
    \ENDIF
    \STATE Update particle Position
    \IF{$p_i(k) < $ position lower bound}
    \STATE $p_i(k) = $ position lower bound
    \STATE $v_i(k) = 0$
    \ELSIF{$p_i(k) > $ position upper bound}
    \STATE $p_i(k) = $ position upper bound
    \STATE $v_i(k) = 0$
    \ENDIF
    \ENDFOR
    \ENDFOR
    \STATE Use $gBest$ as solution
    \end{algorithmic}
    \label{alg:PSOGeneral}

\end{algorithm}

\subsection{Algorithm Challenges}
\label{section:Challenges}


\subsubsection{Premature Swarm Convergence}

\noindent Particle swarm optimization imposes no requirements on the solution search space.
Thus, a variety of local minima may exist, potentially leading to premature swarm convergence on non-optimal solutions. As such, globally optimal solutions are never guaranteed.
This requires running the algorithm many times to increase the probability of finding a near-optimum parameter set. 
This thesis studies a proposed method to limit this premature stagnation.

\subsubsection{Long Computation Time}

\noindent Particle swarm optimization methods require the evaluation of many candidate solutions throughout the algorithm's duration.
Computational complexity in the solution evaluation step is thus amplified, leading to numerically 
expensive algorithms with long execution durations.
Additionally, the many runs required for probabilistic quasi-optimal solution discovery multiplies the overall
algorithm runtime. This limits the use of these stochastic methods 
in time-sensitive calculations. Thus, minimizing algorithm execution time is paramount,
and is chosen as a core study of this paper. \newline

\section{Speedup Test Case Problem Selection} \label{problemSelection}

\noindent To gauge the effectiveness of execution time minimization methods a computationally complex problem is 
necessary. A problem with larger runtimes allows for the effective benchmarking of these reduction strategies,
and presents a non-trivial application of the proposed techniques.\newline

\noindent These criteria make the finite thrust arc problem an apt test case for speedup benchmarking. 
Particles within this problem require lots of execution time for two main reasons

\begin{enumerate}
    \item Each particle requires two separate numerical integrations, one for each thrust arc
    \item Numerical integrations for many potential solutions within the global search space do not converge.
    This requires drastically reduced integration step sizes and massively increases computational complexity. 
\end{enumerate}

\noindent Additionally, this problem consists of a non-trivial 11 unknowns. This broad 11 dimensional search space necessitates 
a comparatively large population
in the search for optimal values. The combination of complex solution evaluation and large swarm sizes make the problem explored within
this thesis sufficient for evaluating potential speedup methods.


\section{Finite Thrust Problem Definition}

\noindent This thesis analyzes the optimal finite thrust transfer problem between two circular orbits as a benchmark case for proposed 
particle swarm optimization improvements. The development of this problem formulation is derived from \citep{Pontani_Conway}. \newline

\noindent This transfer is defined with respect to an initial circular orbit of radius $R_1$ and a final circular orbit of radius $R_2$, subject to the conditions $R_1 > R_2$, where the parameter $\beta = R_2/R_1 > 1$. 
The inertial reference frame is centered at the attracting body. The corresponding coordinate frame definition is as follows: The $x$ axis is aligned with the spacecraft at the initial time $t_0$ and the $y$ axis located in the orbital plane.
The $z$ axis is determined by the right hand rule of these two basis axes. Within this problem $v_r$ denotes the spacecraft radial velocity, $v_\theta$ the horizontal component of velocity, $r$ the radius, and $\zeta$ the angular displacement from the $x$ axis. The gravitational
parameter of the attracting body is $\mu_B$. \newline

\noindent Initial conditions at time $t_0$ are given by
\begin{equation}
v_r(t_0) = 0  \quad  v_\theta(t_0) = \sqrt{\mu_B/R_1} \quad r(t_0) = R_1 \quad \xi(t_0) = 0
\label{initial_conditions}
\end{equation}

\noindent And final conditions given by
\begin{equation}
    v_r(t_f) = 0 \quad v_\theta(t_f) = \sqrt{\mu_B/R_2} \quad r(t_f) = R_2
    \label{final_conditions}
\end{equation}

\noindent The problem assumes a successive transfer trajectory of an initial thrust arc, coasting arc, and second thrust arc. 
Optimization of this problem corresponds to determining the thrust pointing angle function requiring minimum propellant consumption 
subject to the constraints of Eq. (\ref{final_conditions}).\newline

\noindent Additional assumptions are made to simplify the analysis:

\begin{enumerate}
    \item Throughout the duration of each thrust arc maximum thrust is generated
    \item Each thrust arc pointing angle is represented as a third-degree polynomial as a function of time.
\end{enumerate}

\noindent $T$ and $c$ are used to represent the spacecraft thrust level and effective exhaust velocity respectively. The previous assumptions
indicate that the thrust-to-mass ratio has the form

\begin{equation}
\dfrac{T}{m} = \begin{cases} 
    \dfrac{T}{m_0-\frac{T}{c}t} = \dfrac{cn_0}{c-n_0t} & 0\leq t \leq t_1 \\
    0 & t_1\leq t \leq t_2 \\
    \dfrac{T}{m_0-\frac{T}{c}(t_1+t-t_2)} = \dfrac{cn_0}{c-n_0(t_1+t-t_2)} & t_2\leq t \leq t_f 
  \end{cases}
  \label{Toverm}
\end{equation}

\noindent $m_0$ is the initial spacecraft mass and $n_0$ is the initial thrust to mass ratio at $t_0$.
The state space equations of motion for the spacecraft are

\begin{equation}
    \dot{v} = -\dfrac{\mu_B-rv_\theta^2}{r^2}+\dfrac{T}{m}\sin\delta
    \label{vrdot_eom}
\end{equation}
\begin{equation}
\dot{v_\theta} = -\dfrac{v_rv_\theta}{r}+\dfrac{T}{m}\cos\delta
\label{vthetadot_eom}
\end{equation}
\begin{equation}
    \dot{r} = v_r
    \label{rdot_eom}
\end{equation}
\begin{equation}
    \label{xidot_eom}
\dot{\xi} = \dfrac{v_\theta}{r}
\end{equation}
where $\dfrac{T}{m}$ is given in Eq. (\ref{Toverm}) and $\delta$ the thrust pointing angle represented by a
third-order polynomial of time. The state vector used in this problem is $[x_1 \; x_2 \; x_3 \; x_4]^T = [ v_r \; v_\theta \; r \; \xi ]^T$. \linebreak

\noindent The thurst pointing angle $\delta$ is defined as 

\begin{equation}
    \delta = 
    \begin{cases}
        \zeta_0 +\zeta_1t+\zeta_2t^2 +\zeta_3t^3 & 0 \leq t \leq t_1 \\
        \nu_0 + \nu_1(t-t_2) + \nu_2(t-t_2)^2+\nu_3(t-t_3)^3 & t_2 \leq t \leq t_f
    \end{cases}
    \label{delta_eq}
\end{equation} \linebreak

\noindent The optimum thrust pointing angle coefficients $\{\zeta_0 \; \zeta_1 \; \zeta_2 \; \zeta_3 \}$
and $\{\nu_0 \; \nu_1 \; \nu_2 \; \nu_3\}$ are parameters determined by particle swarm optimization. \newline

\noindent During the coasting arc the problem consists of a Keplerian orbit. As such, the semimajor axis $a$ and eccentricity $e$ of the coasting arc can be computed as

\begin{equation}
a = \dfrac{\mu_Br_1}{2\mu_B - r_1(v_{r1}^2+v_{\theta1}^2)}
\label{acoast}
\end{equation}

\begin{equation}
e = \sqrt{1-\dfrac{r_1^2v_{\theta_1}^2}{\mu_Ba}}
\label{ecoast}
\end{equation} \newline

\noindent Provided that the orbit is elliptic ($a > 0$) the true anomaly at $t_1(f_1)$ can be computed as

\begin{equation}
\sin f_1 = \dfrac{v_{r1}}{e}\sqrt{\dfrac{a(1-e^2)}{\mu_B}} \quad \text{and} \quad \cos{f_1} = \dfrac{v_{\theta_1}}{e}\sqrt{\dfrac{a(1-e^2)}{\mu_B}}-\dfrac{1}{e}
\label{trueAnamolyCoast}
\end{equation} \newline

\noindent and the eccentric anomaly $E_1$ as 

\begin{equation}
\sin{E_1} = \dfrac{\sin{f_1}\sqrt{1-e^2}}{1+e\cos{f_1}} \quad \text{and} \quad 
\cos{E_1} = \dfrac{\cos{f_1}+e}{1+e\cos{f_1}}
\label{eccAnamolyCoast}
\end{equation}

\noindent The PSO parameter $\Delta E$ represents the variation in eccentric anomaly througout the coasting arc. Therefore, the eccentric enomaly at $t_2$ is $E_2 = E_1 + \Delta E$. True anamoly can
thus be determined utilizing the results of Eq. (\ref{eccAnamolyCoast})

\begin{equation}
\sin{f_2} = \dfrac{\sin{E_2}\sqrt{1-e^2}}{1-e\cos{E_2}} \quad \text{and} \quad
\cos{f_2} = \dfrac{\cos{E_2}-e}{1-e\cos{E_2}}
\label{trueanamoly2}
\end{equation}

\noindent Furthermore, the coasting time interval $t_{co}$ can be calculated through Kepler's law as
\begin{equation}
t_{co} \overset{\Delta}{=} t_2 - t_1 = \sqrt{\dfrac{a^3}{\mu_B}}[E_2-E_1-e(\sin{E_2}-\sin{E_1})]
\end{equation}

\noindent The results from Eqs. (\ref{acoast}, \ref{ecoast}, and \ref{trueanamoly2}) provide the initial conditions required to numerically integrate 
the spacecraft equations of motion for the second thrust arc beginning at time $t_2$. These initial conditions for the second thrust arc are computed as

\begin{equation}
v_{r_2} = \sqrt{\dfrac{\mu_B}{a(1-e^2)}}e\sin{f_2}
\label{vr2}
\end{equation}

\begin{equation}
    v_{\theta2} = \sqrt{\dfrac{\mu_B}{a(1-e^2)}}(1+e\cos{f_2})
    \label{vtheta2}
\end{equation}

\begin{equation}
r_2 = \dfrac{a(1-e^2)}{1+e\cos{f_2}}
\label{r2}
\end{equation}

\begin{equation}
\xi_2 = \xi_1 + (f_2-f_1)
\end{equation}

\noindent Integrating the second thrust arc with these initial conditions over the given time duration yields the final orbital characteristics. \newline

\noindent This system depends on the eight coefficients representing the thrust pointing angles of the first and second thrust arcs, 
$\{\zeta_0 \; \zeta_1 \; \zeta_2 \; \zeta_3 \}$ and $\{\nu_0 \; \nu_1 \; \nu_2 \; \nu_3 \}$ respectively, and the time intervals for the first thrust arc
$\Delta t_1 \overset{\Delta}{=} t_1$, the coasting arc $\Delta t_{co}$, and the second thrust arc $\Delta t_2 \overset{\Delta}{=} t_f-t_2 $. 
The selection of these 11 unknowns is sought to minimize the objective function 

\begin{equation}
J = \Delta t_1 + \Delta t_2
\label{J}
\end{equation}

\noindent Optimizing the unknown coefficients to minimize $J$ corresponds to the minimization of propellant consumption. Minimizing propellant consumption
results in a maximization of the final-to-initial mass ratio given by 

\begin{equation}
\dfrac{m_f}{m_0} = \dfrac{m_0-\frac{T}{c}\Delta t_1-\frac{T}{c}\Delta t_2}{m_0} = 1-\dfrac{n_0}{c}(\Delta t_1 +\Delta t_2)
\label{finalToInitialMassRatio}
\end{equation}

\noindent The coasting arc is required to be elliptic, therefore any particles with $a \leq 0$ are assigned an infinite objective function value. \newline

\noindent Since $\Delta t_{co}$ can be computed using $\Delta E$, the eccentric anomaly variation replaces the coasting time interval in the 
problem's 11 unknown parameters. Each particle in the swarm thus consists of the following

\begin{equation}
\chi = [ \zeta_0 \quad \zeta_1 \quad \zeta_2 \quad \zeta_3 \quad \nu_0 \quad \nu_1 \quad \nu_2 \quad \nu_3 \quad \Delta t_1 \quad \Delta E \quad \Delta t_2 ]^T
\label{particleUnkowns}
\end{equation}

\noindent To enforce the constraints set forth in Eq. (\ref{final_conditions}) three penalty terms are added to the objective function Eq. (\ref{J}).

\begin{equation}
\tilde{J} = \Delta t_1 + \Delta t_2 + \sum_{k=1}^3 \alpha_k|d_k| 
\label{JwithPenalty}
\end{equation}

\noindent where 

\begin{equation}
d_1 = v_r(t_f) \quad \quad d_2 = v_\theta(t_f) - \sqrt{\dfrac{\mu_B}{R_2}} \quad \quad 
d_3 = r(t_f)-R_2
\label{penaltyValues}
\end{equation}

\noindent A maximum acceptable error of $10^{-3}$ is used and the $\alpha_k$ coefficients assigned as  

\begin{equation}
    \label{penaltyCoefficients}
\alpha_k = \begin{cases}
    100 & |d_k| > 10^{-3} \\
    0 & |d_k| < 10^{-3}
\end{cases}
\end{equation}


\section{Canonical Units Definition}

\noindent This problem employs a canonical set of units to simplify the analysis. One distance unit (DU) is defined as the radius of the initial orbit,
and one time unit (TU) defined such that $\mu_B = 1 \frac{DU^3}{TU^2}$. The unknown coefficients are thus sought within the following ranges

\begin{equation}
\label{particleBounds}
\begin{gathered}
0 \; TU \leq t_1 \leq 3 \; TU \quad 0 \leq \Delta E \leq 2\pi \quad 0 \; TU \leq \Delta t_2 \leq 3 \; TU \\
-1 \leq \xi_k \leq 1 \quad -1 \leq \nu_k \leq 1 \quad (k = 0,1,2,3)
\end{gathered}
\end{equation}

\noindent The effective exhaust velocity $c$ is set to $0.5 \frac{DU}{TU^2}$ and the initial thrust-to-mass ratio set to $0.2 \frac{DU}{TU^2}$. 


\section{Finite Thrust Arc Implementation}

\noindent The finite thrust arc problem explored within this thesis is implemented using the 11 unknown parameters in Eq. (\ref{particleUnkowns})
and corresponding objective function Eq. (\ref{JwithPenalty}). Position and velocity bounds for the particle swarm optimization algorithm are given by Eq. (\ref{particleBounds}). Particle position updating uses 
the standard form

\begin{equation}
    \label{positionUpdating}
p_i = p_i+v_i
\end{equation}

\noindent $i$ ranges from $1$ to the $11$ unknowns. \newline

\noindent Velocity updating is implemented as

\begin{equation}
    \label{eq:velocityUpdating}
v_i = v_ic_i+c_c(pBest_i-p_i)+c_s(gBest-Pi)
\end{equation}

\noindent where the three accelerator coefficients are defined as

\begin{equation}
    \label{eq:acceleratorCoefficients}
c_i = \dfrac{(1+rand(0,1))}{2} \quad c_c = 1.49445(rand(0,1)) \quad c_s = 1.49445(rand(0,1))
\end{equation}

\noindent and $rand(0,1)$ is a uniformly distributed random number generated between 0 and 1. \newline

\noindent A detailed look at the main portion of the PSO pseudocode is shown below

\begin{algorithm}[H]
    \caption{Main Finite Thrust Transfer PSO Algorithm}
    \begin{algorithmic}

    \FOR{ \text{\emph{iteration}} \text{ \textbf{in} \emph{Num Iterations}}}
    \FOR {\text{\emph{particle}} \text{ \textbf{in}  \emph{Swarm}}}
    \STATE Numerically integrate first thrust arc using state-space equations \ref{vrdot_eom},
    \ref{vthetadot_eom}, \ref{rdot_eom}, and \ref{xidot_eom}
    \STATE Compute initial conditions for second thrust arc using equations \ref{acoast}, \ref{ecoast}, and \ref{trueanamoly2}
    \STATE Numerical integrate second thrust arc using state-space equations
    \STATE Evaluate $\tilde{J}$ (\ref{JwithPenalty}) numerically integrated second thrust arc values and penalty terms \ref{penaltyValues} and
    penalty coefficients \ref{penaltyCoefficients}
    \ENDFOR
    \FOR {\text{\emph{particle}} \text{ \textbf{in}  \emph{Swarm}}}
    \STATE Update particle velocity with equation \ref{eq:velocityUpdating}
    \STATE Check velocity bounds
    \STATE Update particle Position with equation \ref{positionUpdating}
    \STATE Check position bounds
    \ENDFOR
    \ENDFOR
    \STATE Use $gBest$ as solution
    \end{algorithmic}
    \label{alg:PSOfThrustMain}

\end{algorithm}


\section{Parallelization}

\noindent The conditions referenced within section \ref{problemSelection} make the finite thrust transfer problem a good candidate for parallelization.
Within each iteration, each particle is evaluated for its value of the objective function $J$. This  contains the computationally heavy numerical integrations. Within this section of the algorithm each particle is evaluated independently. 
As such, parallel processing can be used to evaluate multiple particles simultaneously. Only after each iteration completes and position and velocity updating occurs do the particles become dependent on each others' results. Since 
the swarm updating portion of the algorithm is not computationally expensive, a large potential benefit can be gained from parallelizing the inter-iteration computations. A simplified version of the parallelized algorithm is shown below.

\begin{algorithm}[H]
    \caption{Simplified parallel PSO Pseudocode}
    \begin{algorithmic}
    \FOR{ \text{\emph{iteration}} \text{ \textbf{in} \emph{Num Iterations}}}
    \FOR {\text{\emph{particle}} \text{ \textbf{in}  \emph{Swarm}}}
    \STATE Evaluate objective function $J$
    \hspace{17em}\smash{$\left.\rule{0pt}{1.5\baselineskip}\right\}\ \mbox{in parallel}$}

    \ENDFOR
    \STATE Update particle velocity based on $pBest, gBest$, and current velocity
    \STATE Update particle position
    \ENDFOR
    \end{algorithmic}
    \label{alg:PPSOpsuedocode}
\end{algorithm}


\section{Rehyrdration Definition}

\noindent This thesis also considers the problem of limiting swarm stagnation as mentioned in section (\ref{section:Challenges}).
\textit{Rehydration} is this thesis' term for the random resetting of a percentage of the population under premature convergence.
This method seeks to expand the search space explored throughout the duration of the algorithm. \newline

\noindent Certain stagnation criteria are defined to aid in stagnation determination. To quantify population diversity, the algorithm measures the average
percentage change of $J$, deemed $\delta_{Javg}$, over a pre-defined previous number of iterations, $\eta_{iter}$. The critical 
$\delta_{Javg}$ value for stagnation is pre-defined as $\delta_{Jcrit}$.  If $\delta_{Javg} < \delta_{Jcrit}$, the swarm is considered
stagnated and a portion of the population, $P_{r,\text{\%}}$, is randomly reset. This resetting does not affect the particle's knowledge of its
historical best set of parameters, nor its knowledge of the historically computed global best solution for a given run.\newline

\newpage
