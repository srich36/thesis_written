%%%%%%%%%%%%%%%%%%%%%%%%%%%%%%%%%%%%%%%%%%%%%%%%
% chapter2.tex                                 %
% Contains formatting and content of chapter 2 %
%%%%%%%%%%%%%%%%%%%%%%%%%%%%%%%%%%%%%%%%%%%%%%%%
\chapter{Problem Statement}

\section{Problem Definition}

This thesis analyzes the optimal finite thrust transfer between two circular orbits. A PSO approach is utlized to attempt
to find and optimal solution. The development of this problem is derived from \citep{Pontani_Conway}. \newline

This transfer is developed with respect to an initial circular orbit of radius $R_1$ and a final circular orbit of radius $R_2$ subject to the conditions $R_1 > R_2$ where the parameter $\beta = R_2/R_1 > 1$. 
The inertial reference frame selected for this problem definition is centered at the attracting body. The corresponding coordinate frame definition is as follows: The $x$ axis is aligned with the spacecraft at the initial time $t_0$ and the $y$ axis is located in the orbital plane.
The $z$ axis is determined by the right hand rule of these two basis axis. Within this problem $v_r$ denotes the spacecraft radial velocity, $v_\theta$ the spacecraft the horizontal component of velocity, $r$ the radisu, and $\zeta$ the angular displacement from the $x$ axis. The gravitational
parameter of the attracting body is $\mu_B$. \newline

\noindent Initial conditions at time $t_0$ are given by
\begin{equation}
v_r(t_0) = 0  \quad  v_\theta(t_0) = \sqrt{\mu_B/R_1} \quad r(t_0) = R_1 \quad \xi(t_0) = 0
\label{initial_conditions}
\end{equation}

\noindent And final conditions given by
\begin{equation}
    v_r(t_f) = 0 \quad v_\theta(t_f) = \sqrt{\mu_B/R_2} \quad r(t_f) = R_2
    \label{final_conditions}
\end{equation}

The problem assumes a transfer trajectory of an initial thrust arc, followed by a coasting arc, and ending with a second thrust arc. 
Optimization of this problem corresponds to determining the thrust pointing angle function corresponding to the minimization of propellant consumption 
subject to the constraints of equation \ref{final_conditions}.\newline

\noindent Additional assumptions are made to simplify the analysis:

\begin{enumerate}
    \item Throughout the duration of each thrust arc maximum thrust is generated
    \item Each thrust arc pointing angle is represented as a third-degree polynomial as a function of time.
\end{enumerate}

$T$ and $c$ are used to represent the spacecraft thrust level and effective exhaust velocity respectively. The previous assumptions
indicate that the thrust-to-mass ratio has the following form

\begin{equation}
\dfrac{T}{m} = \begin{cases} 
    \dfrac{T}{m_0-\frac{T}{c}t} = \dfrac{cn_0}{c-n_0t} & 0\leq t \leq t_1 \\
    0 & t_1\leq t \leq t_2 \\
    \dfrac{T}{m_0-\frac{T}{c}(t_1+t-t_2)} = \dfrac{cn_0}{c-n_0(t_1+t-t_2)} & t_2\leq t \leq t_f 
  \end{cases}
  \label{Toverm}
\end{equation}

\noindent Where $m_0$ is the initial spacecraft mass and $n_0$ is the initial thrust to mass ratio at $t_0$.
The state space equations of motion for the spacecraft are as follows

\begin{equation}
    \dot{v} = -\dfrac{\mu_B-rv_\theta^2}{r^2}+\dfrac{T}{m}\sin\delta
    \label{vrdot_eom}
\end{equation}
\begin{equation}
\dot{v_\theta} = -\dfrac{v_rv_\theta}{r}+\dfrac{T}{m}\cos\delta
\label{vthetadot_eom}
\end{equation}
\begin{equation}
    \dot{r} = v_r
    \label{rdot_eom}
\end{equation}
\begin{equation}
\dot{\xi} = \dfrac{v_\theta}{r}
\end{equation}
where $\dfrac{T}{m}$ is given in equation \ref{Toverm} and $\delta$ is the thrust pointing angle represent by a
third-order polynomial of time. The state vector used in this problem is $[x_1 \; x_2 \; x_3 \; x_4]^T = [ v_r \; v_\theta \; r \; \xi ]^T$. \linebreak

\noindent The thurst pointing angle $\delta$ is defined as 

\begin{equation}
    \delta = 
    \begin{cases}
        \zeta_0 +\zeta_1t+\zeta_2t^2 +\zeta_3t^3 & 0 \leq t \leq t_1 \\
        \nu_0 + \nu_1(t-t_2) + \nu_2(t-t_2)^2+\nu_3(t-t_3)^3 & t_2 \leq t \leq t_f
    \end{cases}
    \label{delta_eq}
\end{equation} \linebreak

\noindent Where the optimum thrust pointing angle coefficients $\{\zeta_0 \; \zeta_1 \; \zeta_2 \; \zeta_3 \}$
and $\{\nu_0 \; \nu_1 \; \nu_2 \; \nu_3\}$ are determined by PSO. \newline

During the coasting arc the problem consists of a Kepler orbit. As such, the semimajor axis $a$ and eccentricity $e$ of the coasting arc can be computed

\begin{equation}
a = \dfrac{\mu_Br_1}{2\mu_B - r_1(v_{r1}^2+v_{\theta1}^2)}
\label{acoast}
\end{equation}

\begin{equation}
e = \sqrt{1-\dfrac{r_1^2v_{\theta_1}^2}{\mu_Ba}}
\label{ecoast}
\end{equation} \newline

\noindent Provided that the orbit is elliptic ($a > 0$) the true anomaly at $t_1(f_1)$ can be computed as

\begin{equation}
\sin f_1 = \dfrac{v_{r1}}{e}\sqrt{\dfrac{a(1-e^2)}{\mu_B}} \quad \text{and} \quad \cos{f_1} = \dfrac{v_{\theta_1}}{e}\sqrt{\dfrac{a(1-e^2)}{\mu_B}}-\dfrac{1}{e}
\label{trueAnamolyCoast}
\end{equation} \newline

\noindent and the eccentric anomaly $E_1$ as 

\begin{equation}
\sin{E_1} = \dfrac{\sin{f_1}\sqrt{1-e^2}}{1+e\cos{f_1}} \quad \text{and} \quad 
\cos{E_1} = \dfrac{\cos{f_1}+e}{1+e\cos{f_1}}
\label{eccAnamolyCoast}
\end{equation}

\noindent The PSO parameter $\Delta E$ represents the variation in eccentric anomaly througout the coasting arc. Therefore, the eccentric enomaly at $t_2$ is $E_2 = E_1 + \Delta E$. True anamoly can be
thus determined utilizing the results of equations in \ref{eccAnamolyCoast}

\begin{equation}
\sin{f_2} = \dfrac{\sin{E_2}\sqrt{1-e^2}}{1-e\cos{E_2}} \quad \text{and} \quad
\cos{f_2} = \dfrac{\cos{E_2}-e}{1-e\cos{E_2}}
\label{trueanamoly2}
\end{equation}

\noindent Furthermore, the coasting time interval $t_{co}$ can be calculated through Kepler's law as
\begin{equation}
t_{co} \overset{\Delta}{=} t_2 - t_1 = \sqrt{\dfrac{a^3}{\mu_B}}[E_2-E_1-e(\sin{E_2}-\sin{E_1})]
\end{equation}

\noindent The results from \ref{acoast}, \ref{ecoast}, and \ref{trueanamoly2} provide the initial conditions required to numerical integrate 
the spacecraft equations of motion for the second thrust arc beginning at time $t_2$. These initial conditions for the second thrust arc are computed as

\begin{equation}
v_{r_2} = \sqrt{\dfrac{\mu_B}{a(1-e^2)}}e\sin{f_2}
\label{vr2}
\end{equation}

\begin{equation}
    v_{\theta2} = \sqrt{\dfrac{\mu_B}{a(1-e^2)}}(1+e\cos{f_2})
    \label{vtheta2}
\end{equation}

\begin{equation}
r_2 = \dfrac{a(1-e^2)}{1+e\cos{f_2}}
\label{r2}
\end{equation}

\begin{equation}
\xi_2 = \xi_1 + (f_2-f_1)
\end{equation}

\noindent Integrating the second thrust arc with these initial conditions over the second thrust arc time duration with yield the final orbital characteristics. \newline

This system depends on the eight coefficients representing the thrust pointing angles of the first and second thrust arcs, 
$\{\zeta_0 \; \zeta_1 \; \zeta_2 \; \zeta_3 \}$ and $\{\nu_0 \; \nu_1 \; \nu_2 \; \nu_3 \}$ respectively, and the time intervals for the first coast arc
$\Delta t_1 \overset{\Delta}{=} t_1$, the coasting arc, $\Delta t_{co}$, and the second coast arc $\Delta t_2 \overset{\Delta}{=} t_f-t_2 $. These 11 unknowns are sought to be chosen to minimize the objective function 

\begin{equation}
J = \Delta t_1 + \Delta t_2
\label{J}
\end{equation}

\noindent Optimizing the unknown coefficients to minimize \ref{J} corresponds to the minimization of propellant consumption. Minimizing propellant consumption
results in a maximization of the final-to-initial mass ratio given by 

\begin{equation}
\dfrac{m_f}{m_0} = \dfrac{m_0-\frac{T}{c}\Delta t_1-\frac{T}{c}\Delta t_2}{m_0} = 1-\dfrac{n_0}{c}(\Delta t_1 +\Delta t_2)
\label{finalToInitialMassRatio}
\end{equation}

\noindent The coasting arc is required to be elliptic, therefore any particles with $a \leq 0$ are assigned an infinite value.

Since $\Delta t_{co}$ can be computed using $\Delta E$, the eccentric anomaly variation replaces the coasting time interval in the 
problem's 11 unkown parameters. Each particle in the swarm thus consists of the following

\begin{equation}
\chi = [ \zeta_0 \quad \zeta_1 \quad \zeta_2 \quad \zeta_3 \quad \nu_0 \quad \nu_1 \quad \nu_2 \quad \nu_3 \quad \Delta t_1 \quad \Delta E \quad \Delta t_2 ]^T
\label{particleUnkowns}
\end{equation}


\section{Canonical Units Definition}
\section{Parameter Definition}
\section{Time Optimization (probably needs rewording)}
\newpage
